\documentclass{article}
\usepackage[utf8]{inputenc}
\usepackage{hyperref}

\title{Protokol k projektu MAD}
\author{Maximilián Zivák}
\date{June 2022}

\begin{document}

\maketitle

\section{Subory}
Folder projektu obsahuje
\begin{itemize}
    \item createDB.py - skript ktory spravi databazu z tsv suborov ulozenych v /data
    \item main.py - samotny flask webserver
    \item subfolder /templates
    \begin{itemize}
        \item main.html home page
        \item director.html template pre osobnu stranku kazdeho rezisera
    \end{itemize}
    \item /data
    \begin{itemize}
        \item skript downloadData.py ktory stiahne tsv files z IMDB stranky a unpackne ich
    \end{itemize}
\end{itemize}
\section{Postup}
\begin{itemize}
    \item pip install wget
    \item stiahnem  \href{http://datasets.imdbws.com/}{IMDB datasety} spustenim skriptu v /data
    \item spravim databazu
    \begin{itemize}
        \item v skripte createDB.py zmenim variable maxEntries, defaultne 10e4
        \item spustim skript
    \end{itemize}
    \item spustime webserver pouzitim
    \begin{itemize}
        \item export FLASK\_APP=main.py
        \item flask run --port=PORT (pouzival som 1027 sa mi zda)
    \end{itemize}
    \item otvorim localhost
\end{itemize}
\section{Ine zdroje}
\begin{itemize}
    \item pouzivam open sourcovu \href{https://imdb-api.com/}{imdb api} ktora je limitovana na 100 callov/24hr cez jeden key
\end{itemize}
\section{Linky}
\begin{itemize}
    \item https://imdb-api.com/
    \item https://www.imdb.com/interfaces/ 
    
\end{itemize}
\end{document}
